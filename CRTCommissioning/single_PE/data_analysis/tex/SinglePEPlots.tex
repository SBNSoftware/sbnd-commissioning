\PassOptionsToPackage{usenames,dvipsnames}{color}
\documentclass[11pt]{beamer}
\usetheme{Copenhagen}
\usepackage[utf8x]{inputenc}
\usepackage{color}
\usepackage{ucs}
\usepackage{amsmath}
\usepackage{amsfonts}
\usepackage{amssymb}
\usepackage{array}
\usepackage{booktabs}
%\usepackage{changepage}
%\usepackage{subcaption}
\usepackage{graphicx}
\usepackage{geometry}
%\usepackage{forloop}
\usepackage{pgffor}
%\usepackage{beamerthemesplit}
%\usepackage{comment}
%\usepackage[outdir=./]{epstopdf}
\usepackage{epstopdf}
%\usepackage{wrapfig}
%\usepackage{xintexpr}
%\usepackage[export]{adjustbox}
%\usepackage[orientation=landscape,size=custom,width=16,height=9,scale=0.5,debug]{beamerposter}
%\date{} 
%\subject{}
\usepackage{multicol}
\usepackage{hyperref}

\setbeamertemplate{navigation symbols}{}
\useoutertheme{infolines}


\newcommand\tab[1][1cm]{\hspace*{#1}}
\newcommand{\rulesep}{\unskip\ \vrule\ }
\newcommand{\backupbegin}{
	\newcounter{finalframe}
	\setcounter{finalframe}{\value{framenumber}}
}


%\newcommand{\febs}{193, 87, 88, 85, 80, 81, 79, 89}
\usepackage{pdftexcmds}
\makeatletter
%\newcommand\pdfstrcmp[2] {\pdf@strcmp{#1}{#2}}
\newcommand{\group}{""}
\newcommand{\plot}{""}


\newenvironment{changemargin}[2]{% 
\begin{list}{}{% 
\setlength{\topsep}{0pt}% 
\setlength{\leftmargin}{#1}% 
\setlength{\rightmargin}{#2}% 
\setlength{\listparindent}{\parindent}% 
\setlength{\itemindent}{\parindent}% 
\setlength{\parsep}{\parskip}% 
}% 
\item[]}{\end{list}} 

\makeatletter
\newenvironment{withoutheadline}{
\setbeamertemplate{headline}[default]%evidently this only works for individual frames
\def\beamer@entrycode{\vspace*{-\headheight}}
}{}
\makeatother

\makeatletter
\def\instring#1#2{TT\fi\begingroup
\edef\x{\endgroup\noexpand\in@{#1}{#2}}\x\ifin@}
\makeatother

\setbeamertemplate{headline}
{
}

\newcommand\febs{71,72,75,76,155,153,156,157,158,159,181,182,238}
%\newcommand\febs{134,135,136,137,149,150,151,152}
%\newcommand\febs{71,72,73,75,76,82}

%\newcommand\febs{73, 155}
\newcommand\voltages{154,174,194,214,234}
\newcommand\firstrun{5975}

\begin{document}


%\graphicspath{.}
%\epstopdfsetup{outdir=./}

%\newcommand\runs}{
\section*{}
\begin{frame}{Single PE}
        \begin{multicols}{3} 
               \tableofcontents
        \end{multicols}
\end{frame}

\section{Gain plots}
\begin{frame}
        \Huge{Gain Plots}
\end{frame}

\foreach \feb in \febs{
                \subsection{FEB \feb}

                 \begin{frame}{Module \feb }
                        \centering
                        \foreach \chan in {0, 1, ..., 14}{
                                \IfFileExists{../All_results_start_run\firstrun/gain_vs_bias_voltage_start_run\firstrun/FEB_\feb_channel_\chan.png}{%
                                        \includegraphics[width=0.19\linewidth]{../All_results_start_run\firstrun/gain_vs_bias_voltage_start_run\firstrun/FEB_\feb_channel_\chan}%   
                                }{
                                        \includegraphics[width=0.19\linewidth]{../All_results_start_run\firstrun/raw_gain_vs_bias_voltage_start_run\firstrun/FEB_\feb_channel_\chan}% 
                                }   
                        }
                \end{frame}

                \begin{frame}{Module \feb~cont.}
                        \centering
                        \foreach \chan in {15, 16, ..., 31}{
                                \IfFileExists{../All_results_start_run\firstrun/gain_vs_bias_voltage_start_run\firstrun/FEB_\feb_channel_\chan.png}{%
                                        \includegraphics[width=0.19\linewidth]{../All_results_start_run\firstrun/gain_vs_bias_voltage_start_run\firstrun/FEB_\feb_channel_\chan}%   
                                }{
                                        \includegraphics[width=0.19\linewidth]{../All_results_start_run\firstrun/raw_gain_vs_bias_voltage_start_run\firstrun/FEB_\feb_channel_\chan}% 
                                }    
                        }
                \end{frame}

}

\section{Single PE ADC Plots}

\begin{frame}
        \Huge{Single PE ADC Plots}
\end{frame}


\foreach \voltage in \voltages{
         \subsection{Voltage \voltage}

	\foreach \feb in \febs{
                \subsubsection{FEB \feb}

                \begin{frame}{Voltage \voltage}
                        \centering
                        \includegraphics[width=0.9\linewidth]{../All_results_start_run\firstrun/fitted_adc_histos/fitted_adc_histos_voltage\voltage/gain_vs_channel\feb}%
                \end{frame}

 		\begin{frame}{Voltage \voltage -- Module \feb }
 			\centering
 			\foreach \chan in {0, 1, ..., 14}{
 				\includegraphics[width=0.19\linewidth]{../All_results_start_run\firstrun/fitted_adc_histos/fitted_adc_histos_voltage\voltage/FEB_\feb_channel_\chan_ADC_histo}%
 			}
 		\end{frame}
 		\begin{frame}{Voltage \voltage -- Module \feb~cont.}
 			\centering
 			\foreach \chan in {15, 16, ..., 31}{
 				%\includegraphics[width=0.19\linewidth]{\voltage/adc_feb\feb_chan\chan}%
 				\includegraphics[width=0.19\linewidth]{../All_results_start_run\firstrun/fitted_adc_histos/fitted_adc_histos_voltage\voltage/FEB_\feb_channel_\chan_ADC_histo}%
 			}
 		\end{frame}
 	}

         
%can uncomment the rest of this if we want to add in the chanel by channel adc distributions.
%
 %    \foreach \feb in {193, 87, 88, 85, 81, 80, 79, 89}{
 % %       \begin{frame}{FEB \feb }
  %       \centering
  %       \foreach \chan in {0, 1, ..., 14}{
  %                \includegraphics[width=0.19\linewidth]{../run\run/adc_feb\feb_chan\chan}%
  %       }                 
  %       \end{frame}
  %       \begin{frame}{FEB \feb }
  %       \centering
  %       \foreach \chan in {15, 16, ..., 31}{
  %                %\includegraphics[width=0.19\linewidth]{\run/adc_feb\feb_chan\chan}%
  %              \includegraphics[height=0.2\textheight]{../run\run/adc_feb\feb_chan\chan}%
  %       }
  %       \end{frame}
  %  }        
}   






\end{document}
